\documentclass[a4paper]{article}

\usepackage{amsfonts}
\usepackage[margin=1in]{geometry}

\begin{document}
\large {
Let $\Delta$ be a triangulation of a (pseudo)manifold from which we remove a subset of vertices $W$ together with the induced subcomplex $\Delta_W$. We will anaylize how the number of connected components of the 1-skeleton $G(\Delta)$ on $V\setminus W$ vertices relates to $H_{d-1}(\Delta_W)$. In some cases one completely determines the other, in other cases the way $H_{d-1}(\Delta_W)$ "sits" inside $H_{d-1}(\Delta)$ matters. For pseudomanifolds we obtain that if $H_{d-1}(\Delta_W)$ is trivial removing $W$ does not disconnect the graph and use this to prove lower bounds on connectivity of different classes of complexes. \\

All simplicial complexes we consider will be pure. We will work with homology over $\mathbb{Z}_2 $. \\

Definition: A $d$-dimensional simplcial complex $\Delta$ is a \textbf{weak pseudomanifold} if every $d-1$ face is contained in exactly two $d$ faces. If in addition the link of each face is connected we call $\Delta$ a \textbf{normal pseudomanifold}. A \textbf{pseudomanifold} is a weak pseudomanifold in which the facet graph is connected. It's not hard to show that every normal pesudomanifold is indeed a pseudomanifold.\\

Let $W$ be a subset of the vertex set of $\Delta$. We define the following relation on the facets of $\Delta$ not contained in $W$: $F_1 \sim  F_n$ if there is a sequence of facets $F_1, F_2,...,F_n$ such that $F_i \cap F_{i+1}$ has codimension 1 and is not in $W$. It's easy to see that this is an equivalence relation and we will call the equivalence classes thus obtained \textbf{strong components of $\Delta/W$}. We will denote the number of such classes by \textbf{$S(\Delta/W)$}. \\



\textbf{Theorem 1}: Let $\Delta$ be a d-pseudomanifold and $W$ a subset of vertices. If $H_{d-1}(\Delta_W)=0$ then removing $W$ does not disconnect $G(\Delta)$. \\


Proof: We will in fact prove the stronger statement: $$dim H_0(\Delta_{V-W})\leq S(\Delta/W) \leq dim H_{d-1}(\Delta_W)+1$$

Let $K$ be a strong component in $\Delta/W$. Let $v_1,v_2$ be two vertices contained in  $v_i\in F_i$ for $i=1,2$. We can then build a path between $v_1,v_2$  in $\Delta_{V-W}$ by following the sequence of facets connecting $F_1$ and $F_2$ and at each step choosing a point in $F_i \cap F_{i+1}$ which is not in W. This proves the first inequality. \\


For the second inequality let $K_1,K_2,...,K_n$ be the stong components of $\Delta/W$. These are elements in $C_d(\Delta)$ and $\partial(K_i) \in C_{d-1}(\Delta_W)$ since otherwise one could extend the stong component $K_i$ over a $d-2$ face not in contained in $W$. Since $\partial(\partial(K_i)=0$ we can think of $[\partial(K_i)]$ as elements in $H_{d-1}(\Delta_W)$. Now assume some of the stong components satisfy a linear relation in $H_{d_1}(\Delta_W)$ so their sum will be equal to $\partial(\sigma)$ with $\sigma$ in $C_d(\Delta)$. This implies that 
  $$\partial(\sum K_i+\sigma) =0$$ And this can only happen if all the $K_i$ are in the sum because the top homology of $\Delta$ is supported on the whole complex. Thus we get that any $n-1$ strong components are linearly independent so $S(\Delta/W)= n\leq dim H_{d-1}(\Delta_W)+1  $



\subsection{Applications of Theorem 1}


Let $C(i,d)$ the class of simplicial complexes with $H_d(\Delta)\neq 0$ and 
no missing faces of dimension greater than $i$. These complexes were introduced in [1] by Nevo For given $i,d$  there exist unique integers $0 \leq r$ and $1 \leq r\leq i$ such that $d+1 = qi+r$. Define $$S(i,d) = \partial \sigma^i\star...\star \partial \sigma^i \star \partial \sigma^r $$ where $\partial \sigma^i$, the boundary of the $i$-simplex, appears $q$-times in the join. We have the following lemma.   \\

Lemma (Nevo[1]) If $\Delta$ is in $C(i,d)$ then $f_0(\Delta)\geq f_0(C(i,d))$= q(i+1)+(r+1) = d +1 + q + 1 \\

\textbf{Theorem 2}. Let $\Delta$ be a $d$ - pseudomanifold with no missing faces of dimension higher than i then $G(\Delta)$ is $d+q-1$ - connected. \\ 

Proof: Assume removing the subset of vertices $W$ disconnects $G(\Delta)$. By Theorem 1 we get that $H_{d-1}(\Delta_W)\neq 0$ and since the complex $\Delta_W$ is induced it cannot have any missing faces of dimension higher than $i$. It follows that $\Delta_W \in C(i, d-1)$ so $f_0(\Delta_W) \geq f_0(S(i,d-1) \geq d+q-1 $. \\ \\ 


\textbf{Remark}: Results similar to Theorem 2 have been published in [2] using methods from algebraic topology similar to ours but also commutative algebra.  The results there are for a class of complexes called banner complexes. However we will show that the connectivity bounds we get are tighter. 

We illlustrate this by an example: Let $\Delta = C_3 * C_4$ be the join of 3-cycle and a 4-cycle. This is a 3-sphere. $\Delta$ is not flag since it contains an empty triangle, however it is easily seen that $\Delta \in C(2, 3)$ since it contains no empty tetrahedra by the join construction.  Thus \\


[1] Eran Nevo, \textit{Remarks on Missing Faces and Generalized Lower
Bounds on Face Numbers}





}
\end{document}
