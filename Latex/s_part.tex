\documentclass[a4paper]{article}

\usepackage{amsfonts}
\usepackage{amsthm}
\usepackage{graphicx}
\usepackage[margin=1in]{geometry}

\newtheorem{theorem}{Theorem}[section]
\newtheorem{corollary}{Corollary}[theorem]
\newtheorem{lemma}[theorem]{Lemma}
 

\title{S-Partitions and S-Shellings}
\author{Alexandru Papiu}
\date{October, 23, 2016}
\begin{document}
\large{

\maketitle


Let $\Delta$ be a simplicial complex. We define an \textbf{S-partition} to be an ordering of (not necessarily maximal) faces of $\Delta$: $F_1,F_2,...,F_k$ such that $F_i\cap (F_1\cup...\cup F_{i-1})$ is pure and ($dim F_i -1$)-dimensional and all the facets of $\Delta$ are included in the ordering. If all the faces in the ordering are \textit{facets} we recover the Bjorner-Wachs definition for a non-pure shelling.  $S$-partitions were introduced by Chari in \cite{chari}. \\

Similar to a shelling, an $S$ - partition gives a partition of (the poset) $\Delta$ into intervals $[G_i,F_i]$ such that $\cup [G_i,F_i]$ is a simplicial complex for any $i$. We will call the singleton intervals of the form $[F,F]$ \textbf{critical} faces. \\


Chari showed in \cite{chari_morse} that given an $S$-partition for $\Delta$ one can construct discrete Morse functions on $\Delta$ whose critical faces are exactly the critical faces in the $S$-partition. \\

Note that any simplicial complex admits many $S$-partitions. In particular we have the trivial $S$ - partition into singelton intervals. Clearly this is not a very useful $S$-partition. The basic theme in this note is this: the "coarser" and S-partition is the more information it will gives us about our simplicial complex $\Delta$ both at an algebraic and homological level. \\


Let's also define the \textbf{$h^S$ triangle} as follows: Let $h_{s,i}^S$ to be the number of intervals in $\mathcal{S}$ of the form $[r(F),F]$ such that $|F| = s$ and $|r(F)| = i$. Notice that $$h_i = h_{i,i}+h_{i+1,i}+...+h_{d,i}$$

and $c_i := h_{i,i}$ is the number of critical $i$-cells in $\mathcal{S}$ as well as in the corresponding Morse function $f_\mathcal{S}$. Denote by $c^\mathcal{S}(t) = \sum c_i t^i$. \\

Note that the $h^S$ triangle determines the $f$ vector by the following relation: 
$$f(t) = \sum_{i,j}h_{i,j}t^j(1+t)^{i-j}  $$

We can also express the $h$-vector in terms of the $h^S$ triangle as follows. Using the definition of the $f$ polynomial in terms of the $h$-polynomial we have $$(1+t)^d h(\frac{t}{1+t}) = \sum_{i,j}h_{i,j}t^j(1+t)^{i-j}$$ and by doing a change of variable $\lambda = \frac{t}{1+t}$ we get $$h(\lambda) = \sum_{i,j}h_{i,j} \lambda^j (1-\lambda)^{d-i} $$

Let's quickly introduce some algebraic notation - a good reference is \cite{facesurvey}. \\
	The \textbf{Stanley-Reisner ring} associated to a complex $\Delta$ on vertex set $\{1,2,...,n\}$ is defined as the quotient ring 
$$k[\Delta] = k[x_1,...,x_n]/ I_{\Delta}$$ 
where $I_{\Delta}$ is the ideal generated by the square-free monomials corresponding to the non-faces of $\Delta$. An \textbf{l.s.o.p} is a collection of linear forms $\{ \theta_1,...,\theta_d\}$ in $k[x_1,...,x_n]$,  such that $k[\Delta]/(\theta_1,...,\theta_d)$ is a finite dimensional $k$-vector space. 
We will denote $k[\Delta]/(\theta_1,...,\theta_d)$ by $k(\Delta)$ and call it the \textbf{reduced Stanley-Resiner ring} of $\Delta$ for the specific l.s.o.p we have chosen. \\

We will also need the following technical definitions and result due to Stanley that characterizes l.s.o.p's in terms of a choice function. We are following the presentation in \cite{nonpure2}, section 12. For a set of linear forms $\{\theta_1, \theta_2,...,\theta_n\}$ in $k[\Delta]$ let $M = (m_{i,j})$ be the $d \times n$ matrix defined by $\theta_i = \sum\limits_{j=1}^n m_{i,j}x_j$. Let $F_1,..., F_t$ be the \textit{facets} of $\Delta$ and call a function $C: [t] \rightarrow 2^{[d]}$ a \textbf{nonsingular choice function} if $|C(j)| = |F_j|$ and the square
submatrix with rows in $C(j)$ and columns in $F_j$ is nonsingular, for all facets $F_j$.

\begin{lemma}(Stanley in \cite{stanley}, page 150)
Let $\{\theta_1, \theta_2,...,\theta_n\}$ be a set of linear forms in $k[\Delta]$. Then $\{\theta_1, \theta_2,...,\theta_n\}$ is an l.s.o.p if and only if there exists a non-singular choice function.
\end{lemma}

By Chari's results in \cite{chari_morse} one can use the fundamental theorem of discrete Morse theory to show that the critical faces in an $S$-partitions act as a spanning set for the homology $H_*(\Delta)$. We will show next that a similar results holds true at the algebraic level in $k(\Delta)$ in the following Lemma. One can interpret as a weak version of the Klee-Kleinschmidt Lemma for shellable complexes. \\

 

\begin{lemma} 
Let $\mathcal{S}$ be an S-partition for $\Delta$ with $\Delta$  = $\cup_{i=0}^n [r(F_i), F_i]$. The monomials $\{x^{r(F)}:|r(F)|=i\}$ span $k(\Delta)_i$. \\ 
\end{lemma}

\begin{proof} We will prove this by induction on the number of partitions. If the partition has one element, the restriction will be the empty set and $k(\Delta) = k$ as a $k$-vector space so the lemma is true in this case. \\

Now assume we have added faces $F_1, ..., F_{k-1}$ and now we are adding the interval $[r(F_k), F_k]$. Since $r(F_k)$ is the unique minimal non-face added we get that $$k[\Delta_k]/(x^{r(F_k)}) = k[\Delta_{k-1}]$$ as rings. 

Now let $\theta$ be an l.s.o.p or $\Delta_k$. By Lemma 0.1 above this will also be an l.s.o.p for $\Delta_{k-1}$ so we get that: $$k(\Delta_k)/(x^{r(F_k)}) = k(\Delta_{k-1})$$. \\

Now by the induction hypothesis we have that $\{x^{r(F_1)},..., x^{r(F_{k-1})}\}$ span $k(\Delta)$ so it suffices to show that $x^{r(F_k)}x_i = 0$ in $k(\Delta)$ for any $x_i$. \\

If $i \not\in F_k$ then $\{i\} \cup r(F_k)$ is not a face of $\Delta_k$ so $x^{r(F_k)}x_i = 0$ in $k[\Delta]$ and thus in $k(\Delta)$ as well. \\

Now let's assume $i \in F_k$ and $F_k$ has cardinality $l <d$. By Lemma 0.1 there exits a nonsingular choice function $C$. Now let's select the $C(k)$ rows in the matrix  $M = (m_{i,j})$ defined as above by $\theta_i = \sum\limits_{j=1}^n m_{i,j}x_j$. This gives us a $l \times n$ matrix. Since the $l \times l$ restriction associated with the facet $F_k$ is non-singular we can now use Gaussian elimination to express $x_i$ in terms of the $\theta$'s and monomials not in $F_k$: $$x_i = \sum_{j=1}^l \alpha_j \theta_{C(j)} + \sum_{j \not\in F_k} \beta_j x_j $$ with the $\alpha$'s and $\beta$'s in $k$. When we multiply by $x^{r(F_k)}$ we get both sums on the right to be zero in $k(\Delta)$. Thus $x_i x^{r(F_k)} = 0$ in $k(\Delta)$ and we are done.

\end{proof}


Based on the previous lemma and Chari's result on Morse functions we get the following \textbf{Corollary}: \\

$$c^\mathcal{S}(t) \geq Hilb(H_*(\Delta),k)(t)$$ $$h^\mathcal{S}(t) \geq Hilb(k(\Delta), k)(t) $$ \\

Where $Hilb(H_*(\Delta),k)(t)$ is the Betti polynomial for $\Delta$ over k counting homology ranks and $Hilb(k(\Delta), k)(t)$ is the $\mathbb{Z}$-graded Hilbert series for the reduced Stanley Reisner ring. These inequalities lead naturally to the following definitions: \\

Given a field k. An S-partition is \textbf{k-perfect} if the first inequality is an equality. This is equivalent to saying that the Morse function associated to the S-partition is $k$-perfect. \\

Given a field k and an l.s.o.p $\theta$. An S-partition is an \textbf{$(S,\theta)$ - shelling} if the second inequality is an equality. This is equivalent to saying that the restriction monomials are a basis for the $\mathbb{Z}$-graded reduced Stanley-Reisner Ring. \\

An $S$-partition is \textbf{minimal} if it contains the smallest number of intervals possible. \\

The definition of an $S$-shelling, as it stands is dependent on the system of parameters we choose and this is not a desirable feat - we'd like a more combinatorial interpretation of when an $S$-partition is an $S$-shelling. It turns out that one can get such a characterization for all triangualted manifolds. This is because Schenzel's formula gives the graded dimensions of $k(\Delta)$ in terms of the $h$ vector and homology of $\Delta$. This allows us a clean characterization of an $S$-shelling without having to use the l.s.o.p directly:

\begin{lemma}
Let $\Delta$ be a $d-1$, Buchsbaum complex (this includes all triangulated manifolds with or without boundary) and $\mathcal{S}$ and $S$-partition for $\Delta$. Then $\mathcal{S}$ is an $S$- shelling if and only if it has length $$f_{d-1} + \sum_{i=1}^{d-1} \beta_{i-1}(\Delta) {d-1 \choose i}$$

\end{lemma}

\begin{proof}
By Schenzel's Formula \cite{facesurvey} Theorem 29 we can compute the graded dimensions of $k(\Delta)$ as follows: $$dim_k(k(\Delta)_j) = h_j(\Delta) + {d \choose j} \sum_{i = 1}^{j-1} (-1)^{j-1-i} \beta_{i-1}(\Delta, k)$$. \\

Now adding all the graded parts we get that $k(\Delta)$ has dimension: $$ \sum_{j=0}^d h_j(\Delta) +
\sum_{j=0}^d {d \choose j} \sum_{i = 1}^{j-1} (-1)^{j-1-i} \beta_{i-1}(\Delta, k)$$

The first sum adds to $f_{d-1}$ and the second sum is equal to $$\sum_{i = 1}^{d-1} \beta_{i-1} (\Delta) \sum_{j = i+1}^d {d \choose j} (-1)^{j-i-1} = \sum_{i=1}^{d-1} \beta_{i-1}(\Delta) {d-1 \choose i}$$ 
and the result follows.
\end{proof}

For a 2-manifold Lemma 0.3 tells us that an $S$-partition is an $S$-shelling if and only if it has exactly $f_2 + \beta_1(\Delta, k)$ parts. This means that our $S$-shelling will correspond to adding the facets just like in a shelling plus critical edges, one for each basis cycle in $H_1(\Delta, k)$. \\

\textbf{Question}:A natural question to ask at this point is the following: What is the relationship between a $k$-perfect $S$-partition, minimal $S$-partition, and an $S$-shelling? \\

Note that since the restriction mononials span $k(\Delta)$, an $S$-shelling will always be minimal. However a $k$-perfect $S$-partition need not be minimal. Any collapse of a collabsible complex will give a perfect $S$-parition but these will usually not be minimal since they are partitions containing only intervals of size two. One could try at this point to "consolidate" the 2-partitions to create a coarser $S$-partitions. However as we shall see there are complexes that admit $k$-perfect $S$-partitions but are not $S$-shellable. 


\begin{lemma}
There exist triangulated manifolds that admit $k$-perfect $S$-partitions but do not admit $S$-shellings.
\end{lemma}

\begin{proof}
Notice that if we restrict ourselves to spheres an $S$-partition will be an $S$-shelling if and only if it is a pure shelling. This follows from Lemma 0.3 since a triangulated sphere only has non-trivial homology in the top dimension. 

Also $\Delta$ will admit a perfect $S$-partition if and only if it admits a perfect Morse function. This follows from Chari's result.

So now in order to prove our lemma we have to come up with a triangulated sphere that is perfect but not shellable. Coming up with such examples is not very easy, however in \cite{randommorse}[Section 5.5] Benedetti an Lutz given an examples of a triangulated 3-sphere with a knotted trefoil knot on 3-edges that admits a minimal Morse vector of $(1,0,0,1)$.  However such a sphere cannot be shellable because of the knot.


\end{proof}


\textbf{Question}: Will a $S$-shelling always be $k$-perfect for any $k$?  The answer is yes for $2$-manifolds. \\


\begin{thebibliography}{9}
\bibitem{chari} 
Manoj K. Chari,
\textit{Steiner Complexes, Matroid Ports, and Shellability},
Journal of Combinatorial Theory Series B 59(1). August 1993

\bibitem{chari_morse}
Manoj K. Chari,
\textit{On discrete Morse functions and combinatorial decompositions},
Discrete Mathematics. April 2000

 
\bibitem{facesurvey}
Isabella Novik, Steven Klee,
\textit{Face enumeration on simplicial complexes},
 Recent Trends in Combinatorics. 2016
 
 \bibitem{stanley}
 Richard Stanley,
 \textit{Balanced Cohen-Macaulay complexes}, Trans. Amer. Math. Soc 249 (1979)
 
 \bibitem{nonpure2}
 Bjorner, Wachs,
 \textit{Shellable Nonpure Complexes and Posets}
 
 \bibitem{randommorse}
 Bruno Benedetti, Frank H. Lutz,
\textit{Random Discrete Morse Theory and a New Library of Triangulations},
Experimental Mathematics, Vol. 23, Issue 1 (2014), 66-94

 
\end{thebibliography}

\end{document}